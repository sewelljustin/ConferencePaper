\documentclass[conference]{IEEEtran}
\usepackage{cite}

\begin{document}

\title{Indoor Localization for Mobile Devices Using Bluetooth Low Energy Beacons and Wi-Fi Access Points}

\author{\IEEEauthorblockN{Justin L. Sewell}
\IEEEauthorblockA{School of Electrical and\\Computer Engineering\\
Georgia Institute of Technology\\
Atlanta, Georgia 30332--0250\\
Email: http://www.michaelshell.org/contact.html}
\and
<<<<<<< HEAD
\IEEEauthorblockN{Juan CJ. Morales}
=======
\IEEEauthorblockN{Juan CL. Morales}
>>>>>>> 837d53e3686867c446a54b60c95dda324eec9596
\IEEEauthorblockA{Twentieth Century Fox\\
Springfield, USA\\
Email: homer@thesimpsons.com}}

\maketitle

\begin{abstract}
Lorem ipsum dolor sit amet, consectetur adipiscing elit. Cras dictum vestibulum libero, non ultricies est tempus quis. Cras ultrices lacinia iaculis. Quisque facilisis mi sed libero mollis, dignissim finibus massa porta. Quisque eget fermentum massa. Donec pulvinar quam sapien, non interdum nulla pellentesque vel. Nullam lacinia eros arcu, ut volutpat quam congue eu. Sed eget tincidunt risus. Curabitur sodales rhoncus nibh vitae pellentesque. Quisque urna nisi, maximus eget urna et, faucibus egestas ligula. Maecenas sit amet accumsan tellus. Proin vulputate felis ut risus commodo convallis. Aliquam eu venenatis mi, vitae facilisis arcu. Maecenas faucibus, odio non faucibus tincidunt, erat lectus vestibulum purus, sed vulputate turpis augue a lectus. Vestibulum ante ipsum primis in faucibus orci luctus et ultrices posuere cubilia Curae; Maecenas dolor dui, blandit sit amet nulla egestas, feugiat malesuada quam. Praesent vel posuere libero.\end{abstract}

\section{Introduction}
Indoor navigation systems have been in increasing demand since the introduction of smart-phone technology. The sensors in smart-phones can be used to provide accurate localization in an outdoor environment by using the Global Position System, but so far, no standard indoor localization system has been commercialized.

The problem with such a system pertains to localizing and tracking the user in an indoor space. This issue has many challenges that must be faced that are highlighted in \cite{mainetti2014survey}, as including: the loss of signal precision of wireless systems due to Non-Line-of-Sight (NLOS) conditions and multipath effect, scaling the system for large spaces, complex environments, and the nonstatic nature of persons and obstacles in indoor settings.

A practical, accurate and cost-efficient indoor navigation system that solves these challenges has many beneficial applications such as assisting firemen to navigate a burning, smoke-filled building, locating people in danger in emergency situations, and navigation of public spaces such as malls, airports, and university buildings.

One important but unconsidered application of an indoor navigation system is assistance for the visually impaired. In 2013, there was a reported 7.3 million people in the United States with some form of visual impairment [1]. With no form of eletronic navigation assistance when in an indoor setting, these individuals are hindered when traversing public spaces, such as malls, universities, airports and bus or train stations, among others. This would mean these individuals will need some form of help to locate his or her desired destination in such structures.

Many methods of indoor localization systems have been explored. Previous methods that have been tested use technologies such as GSM (the current global mobile communication standard), radio frequency identification tags (RFID), infrared beacons and receivers, and ultrasonic sensors \cite{otsason2005accurate,li2011performance,liu2014survey,ward1997new,medina2013ultrasound}. Unfortunately, none of these approaches were adopted because of different drawbacks such as short detection ranges, high installation costs, unsuitable levels of accuracy, and little space for improvement.

Other more practical solutions to the localization problem use the Wi-Fi infrastructure that is available in most buildings to reduce cost and installation times. The signal of the Wi-Fi Access Points (AP) can be used to approximate location using the Received Signal Strength Indicator (RSSI). However, These techniques alone are usually not enough to provide acceptable accuracy.

More recently, approaches that use Bluetooth Low Energy (BLE) have been tested. BLE is a technology that has recently surfaced that is used by many devices, including smart-phones. BLE beacons are a great candidate for implementing indoor localization due to their low energy consumption, compact size and affordability.

Recently, Google released Eddystone™, an open BLE beacon format that can be configured to send several different types of payloads using the same packet format \cite{developers.google.com}. Before Eddystone, iBeacon, a proprietary protocol developed by Apple, was the standard format for BLE beacons. Eddystone is much more developer friendly and is becoming very popular due to its compatability with both Android and Apple mobile devices. The format can be used to create a contextually aware experience for users by delivering proximity event-triggered attachments.

Our proposed system, BluNavi, localizes the user by fusing data provided by Inertial Measurement Units (IMUs) and distance approximations calculated from BLE signals. To further increase accuracy, the system is complemented by Wi-Fi fingerprinting , a method which makes use of APs by mapping their RSSI values to absolute locations. Eddystone configured beacons will be used to drive our mobile, context based, indoor navigation application. The system will communicate with the user and the beacons/access points through an Android application to provide accurate, real-time indoor navigation. With this approach we aim to provide a low-cost, widely deployable system while still maintaining  a high-level of accuracy.

The rest of this paper is organized as follows: Section II describes current indoor localization research. Section III explains the methodology behind our approach and section IV contains the evaluation of the experimental results. Lastly, Section V details our conclusions and future work.

\section{Related Work}
Wi-Fi based indoor localization has been a widely researched topic due to its availability, and the recent surge of BLE beacons has also spurred an interest in applying previous methods used in Wi-Fi and other technologies to the advantages of BLE. Most of these approaches use the RSS of the wireless signal to approximate the location of the device.

Wi-Fi Fingerprinting is a highly popular technique in indoor localization \cite{chan2012indoor}\cite{navarro2010wi}. This technique focuses on  building a signal strength map of a given area by creating reference points around it. In each of these reference points, RSSI values are gathered for each available access point found. These values are stored in a database and identified by the reference point in which they were gathered. When a user tries to locate himself, their device will scan for the signal strengths of all available AP’s and match the current values to the ones in the pre-existing database and determine the location of the user. This method brings great advantages due to the fact that the system is fully based on Wi-Fi access points, therefore it incurs no extra hardware costs since it uses the Wi-Fi infrastructure already in any modern public setting. This method of pattern-matching to the signal strength map also eliminates the need of identifying the signal propagation model of the setting in question, which reduces complexity and training time[?]. In addition, correction algorithms such as Nearest Neighbor or the Hidden Markov Model could be applied to the current scans to further improve accuracy.

However, fingerprinting also has it’s shortcomings. The passive scanning of devices, a scan in which the device waits for a broadcast containing the SSID from each AP, increases the time each scan takes, which lowers the rate at which the system can restart the scan and refresh the user’s location. Also, the amount of time per scan can differ depending on the device used, which can further affect the refresh rate.

An improvement on standalone Wi-Fi fingerprinting is examined in \cite{wu2016improved}.  A Particle Filter (PF) is used to fuse location estimations provided by a dead reckoning model and Wi-Fi fingerprinting to provide a higher localization accuracy. The PF is initialized using a Random Sample Consensus model which filters out the outliers of the Wi-Fi fingerprint data by fitting the estimations to the dead reckoning model, thus reducing the chance of the PF initializing in the wrong location. For the fingerprinting, two methods are examined. The first is a probabilistic approach using a Gaussian distribution to approximate the distribution of RSSI values of an AP. The second approach is deterministic using a Support Vector Machine (SVM).  The reported accuracy of the approach was less than 2.9 (m) with an average error distance of 1.2 (m). This approach has good accuracy while not requiring any additional hardware, but it also requires a lengthy off-line training phase for the fingerprint database.

Fingerprinting approaches using the RSSI values from BLE beacons have been tested \cite{faragher2015location}. Using BLE over Wi-Fi has the advantages of faster scan times and lower power consumption. In \cite{faragher2015location}, a grid was established and probabilities were distributed into cells using a Bayesian likelihood function based on the results of a K-Nearest-Neighbor location estimation. Accuracies of less than 2.6m at 90\% of the time were reported with a deployment of 1 beacon per 30m2. This accuracy is better than Wi-Fi fingerprinting and uses less power, however the  system has the drawback of requiring a dense deployment of beacons to achieve medium levels of accuracies.

BLE beacon FP can be combined with a radio frequency propagation model to increase accuracy \cite{zhuang2016smartphone}. The model is built by using the relationship between signal strength and distance. Because of the large levels of noise in the signal strength (RSSI), the relationship is subject to high levels of volatility. Filters, such as an Extended Kalman Filter (EKF), must be applied on the result to achieve a more accurate and stable distance. An outlier detection system can be used on the RSSI data to further improve the filtering process. \cite{zhuang2016smartphone} also uses a polynomial regression model instead of a conventional propagation model to estimate device distance to a beacon. An improved approach to building the radio map by updating the data while the system is online to reduce time of off-line training was also used by this approach. This system achieved distance estimations of less than 2.5m at 90\% of the time for a dense deployment of beacons at 1 beacon per 9m.

\section*{Acknowledgment}


The authors would like to thank...






\bibliographystyle{IEEEtran}
\bibliography{bibliography}
\end{document}
