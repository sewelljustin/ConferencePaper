\documentclass[conference]{IEEEtran}
\usepackage{cite}

\begin{document}

\title{Indoor Localization for Mobile Devices Using Bluetooth Low Energy Beacons and Wi-Fi Access Points}

\author{\IEEEauthorblockN{Justin L. Sewell}
\IEEEauthorblockA{School of Electrical and\\Computer Engineering\\
Georgia Institute of Technology\\
Atlanta, Georgia 30332--0250\\
Email: http://www.michaelshell.org/contact.html}
\and
\IEEEauthorblockN{Juan C. Morales}
\IEEEauthorblockA{Twentieth Century Fox\\
Springfield, USA\\
Email: homer@thesimpsons.com}}

\maketitle

\begin{abstract}
Due to the weaknesses of GPS signals indoors, a successful method of indoor
navigation is an important research topic. An indoor navigation application
running on a mobile device has numerous uses. The application could help new
students locate their classes or professors and it would especially benefit
disabled students by directing the student in the right direction while taking
the shortest route. This project presents a prototype of an indoor navigation
system using Bluetooth Low Energy beacon devices and mobile phones.
Bluetooth Low Energy (BLE) beacons provide a way of locating a mobile device
within doors. The mobile device has the capability to receive Bluetooth signals
sent out by the beacons. A mathematical model is then used to approximate the
distance between the mobile device and the beacon by using the transmitted
power level and received signal strength of the Bluetooth signal. A grid can
then be used to model a floor of a building with a beacon in each quadrant.
Methods of increasing the accuracy of the application include using a density
algorithm on a sample of received signals and also a probabilistic algorithm
that implements a variation of Bayes theorem.
\end{abstract}

\section{Introduction}
There are more than 7.3 million people in the United States with some form of visual impairment [1]. For many of these individuals, navigation has always been a challenge. The ease of navigation for the blind was greatly improved with the introduction of the Global Positioning System (GPS) and the widespread deployment of systems that use the GPS. However, due to the GPS satellites’ line of sight requirements, these individuals are left unassisted upon entering an indoor environment because no practical system for indoor navigation has been introduced.

The problem with such a system pertains to localizing and tracking the user in an indoor space. This issue has many challenges that must be faced that are highlight in [2], as including: the loss of signal precision of wireless systems due to Non-Line-of-Sight (NLOS) conditions and multipath effect, scaling the system for large spaces, complex environments, and the nonstatic nature of persons and obstacles in indoor settings.

A practical, accurate and cost-efficient indoor navigation system that considers these challenges would greatly benefit the visually impaired. With no form of navigation assistance when in an indoor setting, these individuals are hindered when traversing public spaces, such as malls, universities, airports and bus or train stations, among others. This would mean these individuals will need some form of help to locate his or her desired destination in such structures. An option for such assistance is a navigation application for a mobile device. Such an application would require the user to be localized within an indoor space.

Many methods of indoor localization systems have been explored, using a wide range of hardware based techniques. Previous methods that have been tested use technologies such as GSM (the current global mobile communication standard), radio frequency identification tags (RFID), infrared beacons and receivers, and ultrasonic sensors [3] - [7]. These approaches suffer from various drawbacks such as short detection ranges, high installation costs, unsuitable levels of accuracy, and little space for improvement.

Other more practical solutions to the localization problem use the already in place Wi-Fi infrastructure of buildings to reduce both cost and installation times while maintaining efficient levels of accuracy. Wi-Fi Access points communicate with other devices via 802.11 standards, and the signal strength being received by any device from a Wi-Fi access point, known as the Received Signal Strength Indicator, or RSSI, can be retrieved and used through various techniques to achieve Indoor Localization. [?][Might still have the same issue..]

More recently, approaches that use Bluetooth Low Energy (BLE) beacons have been examined. BLE is a technology that has recently surfaced that is a great candidate for implementing indoor localization due to its low energy consumption, affordability and compact size. Unlike classic Bluetooth, BLE scan times are much shorter, allowing for a quicker refresh rate. In 2015, Google released Eddystone™, an open BLE beacon format that can be configured to send several different types of payloads using the same packet format [8]. [BEFORE EDDYSTONE, THE ONLY FORMAT WAS iBEACON AND ALTBEACON...] The format can be used to create a contextually aware experience for users by delivering proximity event-triggered attachments. Other formats such as iBeacon do not provide this functionality. Eddystone is also more developer friendly and formats can be configured based on context of application.[EDDYSTONE IS MORE DEVELOPER FRIENDLY THAN PAST FORMATS - GIVE FORMAT WEIGHT]

BluNavi localizes the user by fusing data provided by Inertial Measurement Units (IMUs) and distance approximations of the BLE beacons. To further increase the accuracy, the system is complemented by Wi-Fi fingerprinting , a method which makes use of access points by mapping their RSSI values to absolute locations. Eddystone™ configured beacons will be used to drive our mobile, context based, indoor navigation application for the visually impaired. The system will communicate with the user and the beacons/access points through an Android application to provide accurate, real-time indoor navigation. With this approach we aim to provide a low-cost, widely deployable system while still maintaining  a high-level of accuracy.

The rest of this paper is organized as follows: Section II describes current indoor localization research. Section III explains the methodology behind our approach and section IV contains the evaluation of the experimental results. Lastly, Section V details our conclusions and future work.

\section{Related Work}
[INTRO PARAGRAPH]
Wi-Fi based indoor localization has been a widely researched topic due to its availability, and the recent surge of BLE beacons has also spurred an interest in applying previous methods used in Wi-Fi and other technologies to the advantages of BLE. Most of these approaches use the RSS of the wireless signal to approximate the location of the device.

[Wi-Fi Fingerprinting]

Wi-Fi Fingerprinting is a highly popular technique in indoor localization [9][10]. This technique focuses on  building a signal strength map of a given area by creating reference points around it. In each of these reference points, RSSI values are gathered for each available access point found. These values are stored in a database and identified by the reference point in which they were gathered. When a user tries to locate himself, their device will scan for the signal strengths of all available AP’s and match the current values to the ones in the pre-existing database and determine the location of the user. This method brings great advantages due to the fact that the system is fully based on Wi-Fi access points, therefore it incurs no extra hardware costs since it uses the Wi-Fi infrastructure already in any modern public setting. This method of pattern-matching to the signal strength map also eliminates the need of identifying the signal propagation model of the setting in question, which reduces complexity and training time[?]. In addition, correction algorithms such as Nearest Neighbor or the Hidden Markov Model could be applied to the current scans to further improve accuracy.

However, fingerprinting also has it’s shortcomings. The passive scanning of devices, a scan in which the device waits for a broadcast containing the SSID from each AP, increases the time each scan takes, which lowers the rate at which the system can restart the scan and refresh the user’s location. Also, the amount of time per scan can differ depending on the device used, which can further affect the refresh rate.

WiFi fingerprinting with PDR (paper sent by Dr. Labrador)
An improvement on standalone Wi-Fi fingerprinting is examined in [11].  A Particle Filter (PF) is used to fuse location estimations provided by a dead reckoning model and Wi-Fi fingerprinting to provide a higher localization accuracy. The PF is initialized using a Random Sample Consensus model which filters out the outliers of the Wi-Fi fingerprint data by fitting the estimations to the dead reckoning model, thus reducing the chance of the PF initializing in the wrong location. For the fingerprinting, two methods are examined. The first is a probabilistic approach using a Gaussian distribution to approximate the distribution of RSSI values of an AP. The second approach is deterministic using a Support Vector Machine (SVM).  The reported accuracy of the approach was less than 2.9 (m) with an average error distance of 1.2 (m). This approach has good accuracy while not requiring any additional hardware, but it also requires a lengthy off-line training phase for the fingerprint database.

Fingerprinting with BLE beacons
Fingerprinting approaches using the RSSI values from BLE beacons have been tested [12]. Using BLE over Wi-Fi has the advantages of faster scan times and lower power consumption. In [12], a grid was established and probabilities were distributed into cells using a Bayesian likelihood function based on the results of a K-Nearest-Neighbor location estimation. Accuracies of less than 2.6m at 90\% of the time were reported with a deployment of 1 beacon per 30m2. This accuracy is better than Wi-Fi fingerprinting and uses less power, however the  system has the drawback of requiring a dense deployment of beacons to achieve medium levels of accuracies.

Smartphone-Based Indoor Localization with BLE beacons
BLE beacon FP can be combined with a radio frequency propagation model to increase accuracy [13]. The model is built by using the relationship between signal strength and distance. Because of the large levels of noise in the signal strength (RSSI), the relationship is subject to high levels of volatility. Filters, such as an Extended Kalman Filter (EKF), must be applied on the result to achieve a more accurate and stable distance. An outlier detection system can be used on the RSSI data to further improve the filtering process. [13] also uses a polynomial regression model instead of a conventional propagation model to estimate device distance to a beacon. An improved approach to building the radio map by updating the data while the system is online to reduce time of off-line training was also used by this approach. This system achieved distance estimations of less than 2.5m at 90\% of the time for a dense deployment of beacons at 1 beacon per 9m.

\section*{Acknowledgment}


The authors would like to thank...


\begin{thebibliography}{1}

\bibitem{IEEEhowto:kopka}
H.~Kopka and P.~W. Daly, \emph{A Guide to \LaTeX}, 3rd~ed.\hskip 1em plus
  0.5em minus 0.4em\relax Harlow, England: Addison-Wesley, 1999.

\end{thebibliography}




% that's all folks
\end{document}
